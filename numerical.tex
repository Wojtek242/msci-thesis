\section{Numerical Simulations}\label{sec:numerical}

\subsection{The Algorithm}

We solve the two-particle Schr\"{o}dinger equation for the Hamiltonian
\eqref{eq:main} via the explicit staggered method \cite{numerics}. We
choose our units such that $\hbar = 1$ and the neutron mass $m_n =
1$. The wave function is evaluated on a grid of discrete values for
the independent variables
\begin{equation}
\Psi(x_1, x_2, t) = \Psi(x_1 = l\Delta x, x_2 = m\Delta x, t = n\Delta
t) \equiv \Psi^n_{l, m},
\end{equation}
where $l$, $m$ and $n$ are integers. We separate it into real and
imaginary parts,
\begin{equation}
\Psi^n_{l,m} = u^{n+1}_{l,m} + iv^{n+1}_{l,m},
\end{equation}
which allows us to solve the Schr\"{o}dinger equation via a finite
difference method with a pair of coupled equations:
\begin{equation}
\begin{array} {lcl}
u^{n+1}_{l,m} & = & u^{n-1}_{l,m} - \left\{ \alpha \left[ v^n_{l+1,m}
  + v^n_{l-1,m} + v^n_{l,m+1} + v^n_{l,m-1} - 4v^n_{l,m} \right] - 2
\Delta t V_{l,m}v^n_{l,m} \right\} ,
\end{array}
\end{equation}
\begin{equation}
\begin{array} {lcl}
v^{n+1}_{l,m} & = & v^{n-1}_{l,m} + \left\{ \alpha \left[ u^n_{l+1,m}
  + u^n_{l-1,m} + u^n_{l,m+1} + u^n_{l,m-1} - 4u^n_{l,m} \right] - 2
\Delta t V_{l,m}u^n_{l,m} \right\},
\end{array}
\end{equation}
where 
\begin{equation}
\alpha = g^{xx}\frac{\Delta t}{\Delta x^2},
\end{equation}
and $V_{l,m}$ is the combined value of the potential due to the
harmonic well and neutron-neutron interaction on the discretised
grid given by
\begin{equation}
V_{l,m} = \frac{1}{2}\omega^2(l^2 + m^2)\Delta x^2 + \nu\delta_{l,m}.
\end{equation}
In order to evaluate the value of the real part at step $n+1$ we need
to know the values of the real part at $n-1$ and the imaginary part at
$n$. The same is true for evaluating the imaginary part. Therefore, we
do not have to calculate both parts at every time step and we compute
the real and imaginary parts at slightly different (staggered)
times. The form of the discretised equations is identical to those
presentied in \cite{numerics}. However, the value of $\alpha$ is now
scaled by the oscillating metric tensor component $g^{xx}$. This novel
modification does not lead to instabilities and simulations have been
confirmed to conserve probability to the same precision as before.

For the more general Laplacian \eqref{eq:general}, the equations have to
be modified even further and new terms are introduced for the
wavefunction's first derivative terms present. This too, does not
cause instabilities and conserves probability very well.
