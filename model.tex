\section{The Model}\label{sec:model}

\subsection{Gravitational Waves}

Any reasonable signal possible to detect on Earth will be of
astrophysical origin. Hence, we can assume that the source is far
enough to treat the waves as plane waves. Therefore, in order for our
model to be able to account for several gravitational waves we have to
expand the formalism for the plane wave solutions presented in
appendix \ref{sec:appendix} to a superposition of multiple waves. We
adopt the viewpoint that $h_{\mu\nu}$ is a symmetric tensor field
(under global Lorentz transformations) defined in quasi-Cartesian
coordinates on a flat Minkowski background spacetime. Since we work
with linearised general relativity we can easily obtain the solution
by superposing the single plane wave solutions
\begin{equation}\label{eq:manywaves}
\bar{h}^{\mu\nu} = \sum_j(A_j)^{\mu\nu}\exp(i(k_j)_\rho x^\rho),
\end{equation}
where $A_j^{\mu\nu}$ are constant components of symmetric tensors and
$(k_j)_\mu$ are the constant, real components of vectors. We assume the
summation convention, but explicitly state the summation over the
index $j$ as it does not run over the coordinate indices, but over all
plane waves present.

We consider a guage transformation of the same form as the transverse
traceless gauge used in appendix A which is defined as
\begin{equation}\label{eq:manyTT}
\bar{h}^{0a}_{TT} \equiv 0 \,\,\,\,\,\, \text{and} \,\,\,\,\,\, \bar{h}_{TT} \equiv 0,
\end{equation}
where latin alphabet indices run over the spatial dimensions
only. Furthermore the Lorenz gauge condition gives the constraints
\begin{equation}
\partial_0\bar{h}^{00}_{TT} = 0 \,\,\,\,\,\, \text{and} \,\,\,\,\,\, \partial_a\bar{h}^{ab}_{TT} = 0.
\end{equation}
Whilst we cannot consider this gauge to be transverse anymore since we
are considering waves travelling in different directions, we will keep
the labels since we are using exactly the same definition. This
generalisation is straightforward, because of the linear nature of the
field equations in a weak gravitational field. The field tensor
transforms as 
\begin{equation}\label{eq:transformation}
\bar{h}^{\mu\nu}_{TT} = \bar{h}^{\mu\nu} - \partial^{\mu}\xi^{\nu} -
\partial^{\nu}\xi^{\mu} + \eta^{\mu\nu}\partial_\rho\xi^\rho,
\end{equation}
where $\xi^\mu$ are four functions that define the gauge
transformation and which must satisfy $\Box^2\xi^\mu$ to preserve the
Lorenz gauge. The solution in \eqref{eq:manywaves} is a linear
superposition of waves with different wavevectors and Each of the
components can be transformed into the TT gauge using some set of
functions $\xi^\mu_j$ (see appendix \ref{sec:appendix} for
details). Therefore, we can transform \eqref{eq:manywaves} using
$\xi^{\prime\mu} = \sum_j \xi^\mu_j$ since the transformation
\eqref{eq:transformation} is linear in $\xi^\mu$. Applying this
transformation gives us the same constraints on the constant
$A_j^{\mu\nu}$ components separately for all values of $j$. Therefore
in the new gauge the coefficients must satisfy
\begin{equation}
(A_j)^{0a}_{TT} = 0 \,\,\,\,\,\, \text{and} \,\,\,\,\,\, ((A_j)_{TT})^\mu_\mu = 0
\end{equation}
and the Lorenz gauge conditions require that
\begin{equation}
(A_j)^{00}_{TT} = 0 \,\,\,\,\,\, \text{and} \,\,\,\,\,\, (A_j)^{ab}_{TT}k_b = 0.
\end{equation}
Using these conditions we can construct a traceless tensor which
satisfies the definition in \eqref{eq:manyTT}. It is a superposition
of waves of the form \eqref{eq:manywaves}
\begin{equation}\label{eq:manyh}
\bar{h}^{\mu\nu}_{TT} = \sum_j(A_j)_{TT}^{\mu\nu}\exp(i(k_j)_\rho x^\rho),
\end{equation}
where $(A_j)_{TT}^{0\mu} = (A_j)_{TT}^{\mu 0} = 0$, the spatial
components are given by
\begin{equation}
(A_j)^{ab}_{TT} = ((P_j)^a_c(P_j)^b_d - \frac{1}{2}(P_j)^{ab}(P_j)_{cd})(A_j)^{cd},
\end{equation}
$(P_j)_{ab} \equiv \delta_{ab} - (n_j)_a(n_j)_b$ and $(n_j)_a =
(\hat{k}_j)_a$.

\subsection{Neutron-neutron Interaction}

Nucleon-nucleon scattering is fundamentally a many body problem
governed by quantum chromodynamics. However, the strong interaction
has a very short range ($\sim$ a few fm) and $kR \ll 1$, where $k$ is
the neutron wave vector and $R$ the range of the potential. This means
that we only have to consider s-wave scattering and we can approximate
the scattering potential with a delta function $V_{nn} =
U\delta(\vec{r}_1 - \vec{r}_2)$. To obtain the value of $U$ we use the
fact that for s-wave scattering the cross-section is given by $\sigma
= 4\pi a_s^2$, where $a_s$ is the scattering length which can be
measured experimentally. Therefore, in the Born approximation we
obtain
\begin{equation}\label{eq:nn}
U = \frac{2 \pi a_s \hbar^2}{m_n},
\end{equation}
where $m_n$ is the mass of the neutron. Gravitational waves cause the
distance between the two neutrons, $|\vec{r}_1 - \vec{r}_2|$, to
oscillate. However, because the inter-particle interaction is in the
form of a contact potential it remains unaffected in the presence of a
passing wave. The physical spatial separation will only be zero if the
coordinate separation vector will also be zero.

\subsection{The Schr\"{o}dinger Equation}

The gravitational wave background is predicted to have no significant
components at wavelengths shorter than a few km which is much larger
than the lengthscales we are considering for the
experiment. Therefore, combined with the fact that we are only
considering weak gravitational fields, we can assume that in order to
model the effect of gravitational waves on the quantum state of two
neutrons in a harmonic trap we do not need to resort to a full quantum
description of the interaction.

In order to incorporate the effect of passing waves on the quantum
state we begin by considering the two-particle Schr\"{o}dinger
equation for the Hamiltonian given in \eqref{eq:hamiltonian}
\begin{equation}\label{eq:schrodinger}
i\hbar\frac{\partial \Psi(x_1, x_2; t)}{\partial t} = \left[
  -\frac{\hbar^2}{2m}\left( \frac{\partial^2}{\partial x_1^2} +
  \frac{\partial^2}{\partial x_2^2} \right) +
  \frac{1}{2}m\omega^2(x_1^2 + x_2^2) + V_{nn}(x_1-x_2)
  \right]\Psi(x_1, x_2; t).
\end{equation}
We have already shown that the form of $V_{nn}$ is unaffected. The
potential due to the interaction with the harmonic trap also remains
unchanged as it does not depend on the physical separation between two
points. A huge benefit of the gauge we have chosen to work in is the
fact that only the spatial components of the metric are affected by
the gravitational wave which means that the time derivative on the
left hand side does not have to be modified. Only the spatial
derivatives have to be considered in this situation.

For a general metric the Laplacian of a scalar field is given by
\begin{equation}\label{eq:laplacian}
\nabla^2\phi = \frac{1}{\sqrt{|g|}}\partial_a \left( \sqrt{|g|} g^{ab}
\partial_b\phi \right),
\end{equation}
where $g$ is the determinant of the metric tensor which in the TT
gauge, to first order in $h$, is equal to unity. We now want to reduce
the problem to only one spatial dimension to simplify the model. As
this is only an investigation into the feasibility of such an
experiment such a simplification is desirable as it reduces the
necessary computational time required to obtain qualitatively the same
results. The gravitational waves are very weak so we assume that
confinement in the $y$ and $z$ directions is strong enough that the
wavefunction remains in its ground state along those axes at all
times. This means that we can ignore all second derivative terms apart
from $\frac{\partial^2 \Psi}{\partial x^2}$ since this is the only
significant kinetic energy term. Furthermore, we will have terms of
the form $\partial_a g^{ab} \partial_b \phi$. These terms can also be
ignored, because $\partial_a g^{ab} \propto k_a$ and for any realistic
setup the wavelength of the waves will be much larger than the size of
the expriment making these terms insignificant. Therefore the only
relevant terms that we are left with are
\begin{equation}\label{eq:kinetic}
\nabla^2 \Psi = g^{xx}\frac{\partial^2 \Psi}{\partial x^2},
\end{equation}
where the effective form of $g^{xx}$ is obtained from \eqref{eq:manyh}
\begin{equation}
g^{xx} = 1 + \sum_i A^{xx}_i \cos(\Omega_i t + \phi_i),
\end{equation}
$\Omega_i$ is the frequency of the $i$th wave, $\phi_i$ is the $i$th
wave's phase at $x=0$ and we have ignored phase variation along the
trap axis, $k_x x$, since we consider wavelengths much larger than the
dimensions of the well.

The most general form of the one-dimensional Laplacian in the TT gauge
that preserves probability when used in the Sch\"{o}dinger equation is
\begin{equation}\label{eq:general}
\nabla^2 \Psi = g^{xx}\frac{\partial^2 \Psi}{\partial x^2} +
\frac{\partial g^{xx}}{\partial x} \frac{\partial \Psi}{\partial x}.
\end{equation}
This imposes some additional restrictions on the components of the
metric tensor. We again ignore second derivative terms due to
confinement along the other axes. However, we must now require $g^{xy}
= g^{xz} = 0$ for equation \eqref{eq:general} to be correct. In order
to investigate gravitational waves with shorter wavelengths we want to
be able to use this equation for the Laplacian. However, under the TT
and Lorenz gauge conditions the additional constraints also require
$g^{xx} = 0$ unless the wavevectors are perpendicular to the
$x$-axis. This in turn implies $\partial_x g^{xx} = 0$ since $k_x =
0$. Therefore, in our investigation we always use equation
\eqref{eq:kinetic}, but for wavelengths comparable to or smaller than
the trap dimensions this limits us to the study of waves perpendicular
to the $x$-axis.

Neutrons are spin-$\frac{1}{2}$ particles and so also we have to
address the question of how the intrinsic angular momentum of the
particles couples to the gravitational waves. The theoretical
possibility of such coupling was first considered by Kobzarev and Okun
in 1963 \cite{kobzarev}. If we consider a Newtonian gravitational
potential $\phi$ then the coupling to spin would take the form
$H_{int}=A\vec{\sigma}\cdot\nabla\phi$, where $A$ is an amplitude
and $\vec{\sigma}$ is the particle spin. However, this violates the
equivalence principle which states that the trajectory of a point mass
in a gravitational field depends only on its initial position and
velocity, and is independent of its composition. Therefore, we do not
have to take into account any spin-gravity coupling since we are
working in the weak field limit.

The final form of the two-particle Hamiltonian that we
will be investigating is
\begin{equation}\label{eq:main}
\hat{H} = - g^{xx} \frac{\hbar^2}{2m} \left(
  \frac{\partial^2}{\partial x_1^2} + \frac{\partial^2}{\partial
    x_2^2} \right) + \frac{1}{2}m\omega^2(x_1^2 + x_2^2) +
V_{nn}(x_1-x_2).
\end{equation}

