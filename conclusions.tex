\section{Conclusions}\label{sec:conclusions}

We have proposed and investigated the feasibility of an experiment to
detect gravitational waves using the entanglement of two neutrons
trapped in a harmonic well. The quantum dynamics of the two particles
lead to entanglement which is affected by the presence of
gravitational radiation. We have shown that entanglement amplifies the
effect of high frequency gravitational waves on the
wavefunction. However, for realistic values of the wave amplitudes the
effect is too small to be measured in a device with the dimensions of
a typical multilayer.

The effects of gravitational waves were combined with the quantum
dynamics of the two neutrons in the weak-field limit, where the
linearised field equation could be used. The effects of the
oscillating metric were combined with the Schr\"{o}dinger equation by
modifying the kinetic energy term. The potential due to the harmonic
well and the inter-nucleon interaction remained unchanged.

Numerical solutions to the modified time-dependent Schr\"{o}dinger
equation were obtained with the explicit staggered method. It was
shown that there are two different behaviour regimes. For low
frequency waves there are no additional quantum effects and the
difference in the final quantum state is due to the different
classical particle trajectories. These waves were not investigated as
there are better ways of detecting them. However, the high frequency
regime couples with the particle interaction and the effect is
strongly dependent on its strength and is maximised close to the value
for which the particles maximally entangle. This is an interesting
result as it is a possible mechanism for detecting high frequency
gravitational waves. However, for any experimentally accessible values
we would not observe anything as the neutron-neutron interaction is
too strong for any entanglement to be generated via the system's
dynamics alone.

This experiment is not solely limited by the size of the signal. One
other issue that would be difficult to resolve when building such an
experiment is the isolation of the system from all environmental
effects except for the gravitational waves. We have shown that the
effect of radiation on the quantum state is extremely small and that
entanglement is a key element, but with current technologies it is
difficult to even maintain entangled resources for times longer than a
few nanoseconds \cite{decoherence} unless we are working with trapped
particles in ultra-high vacuum. Limiting undesirable decoherence is
difficult with current technology and so any effects due to
gravitational waves would be unobservable.
