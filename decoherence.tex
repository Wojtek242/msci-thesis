\section{Decoherence}\label{sec:decoherence}

Decoherence is the process through which quantum correlations disperse
information throughout the degrees of freedom that are effectively
beyond the observer's control. It has been suggested as the mechanism
through which a measurement of a quantum system is effectively made
\cite{zurek}. In general, any interaction with the environment will
eventually lead to decoherence and this is one of the main obstacles
in generating and maintaining entangled resources. At the beginning of
this investigation we proposed that entanglement may be sensitive
enough to be affected by gravitational waves. In our approach we have
only investigated the dynamics of two neutrons in a harmonic well
subject to gravitational radiation by solving the Schr\"{o}dinger
equation for the modified Hamiltonian \eqref{eq:main}. This has
allowed us to study the effects of the waves on the wavefunction of
neutrons entangling in a harmonic potential, but the fact that we keep
track of all of the degrees of freedom meant that it was impossible to
observe decoherence.

In order to be able to describe decoherence we have to construct a
dynamical model of a system coupled to its environment. In the
Feynman-Vernon theory of the influence functional \cite{feynman} we
consider a system A interacting with a second system B (the reservoir)
described by the Hamiltonian
\begin{equation}
H = H_A + H_I + H_B,
\end{equation}
where system A consists of a single particle, the reservior consists of
N particles and $H_I$ is the interaction Hamiltonian. We can obtain
the reduced density operator for system A by tracing out all the
environment coordinates. Assuming that the initial density operator is
separable into a product of the density operators for A and B we get
\begin{equation}
\rho(x, y, t) = \int \mathrm{d} x^\prime \mathrm{d} y^\prime J(x, y,
t; x^\prime, y^\prime, 0)\rho_A(x^\prime, y^\prime, 0),
\end{equation}
where x and y are the position coordinates of the single particle,
\begin{equation}
J(x, y, t; x^\prime, y^\prime, 0) = \int \int \mathrm{D} x \mathrm{D}
y \exp \left[ i \frac{S_A[x]}{\hbar} \right] \exp \left[ -i
  \frac{S_A[y]}{\hbar} \right] F[x, y],
\end{equation}
$F[x, y]$ is the so called influence functional given by
\begin{equation}
\begin{array} {lcl}
F[x, y] & = & \int \mathrm{d} \bm{R^\prime} \mathrm{d} \bm{Q^\prime}
\mathrm{d} \bm{R} \rho_B(\bm{R^\prime}, \bm{Q^\prime}, 0) \int \int
\mathrm{D} \bm{R} \mathrm{D} \bm{Q} \\\\ & & \times \exp \left[
  \frac{i}{\hbar} \left( S_I[x, \bm{R}] - S_I[y, \bm{Q}] + S_B[\bm{R}]
  - S_B[\bm{Q}] \right) \right],
\end{array}
\end{equation}
$S_i$ is the action corresponding to the Hamiltonian $H_i$ and
$\bm{R}$, $\bm{Q}$ are vectors of the positions of the reservoir
particles. This has been solved exactly in \cite{feynman} in the limit
of a weak coupling, where we only have to consider the linear response
of the reservoir to the system and we can describe the environment in
the harmonic approximation.

The gravitational wave background could potentially be modelled with a
collection of harmonic oscillators. However, as we have seen in
section 3 the interaction of the wave function with gravitational
waves is implemented through a modification of the kinetic energy term
and not via an effective potential and modelling the coupling with the
system with a linear response would be insufficient. The more
complicated interaction would lead to an intractable form for the
influence functional. Furthermore, there is no way of obtaining an
irreversible process such as decoherence using the above solution. We
do not consider a reservoir of infinite size and any information lost
by the system will return within a finite period of time.

Caldeira and Leggett addressed the issue of irreversibility in 1983
\cite{caldeira} in an attempt to solve the problem of quantum Brownian
motion. Instead of coupling the system A to a reservoir B they
replaced the system-environment coupling with an externally applied
classical force $F(t)$. The fluctuating force has to obey the standard
stochastic relations
\begin{equation}
\langle F(t) \rangle = 0,
\end{equation}
\begin{equation}
\langle F(t) F(t^\prime) \rangle = 2 \eta k_B T \delta (t - t^\prime),
\end{equation}
where $\eta$ is a damping constant, $k_B$ is Boltzmann's constant and
$T$ is the temperature. However, there are issues with applying the
Caldeira-Leggett model to our problem regarding the nature of the
effect produced by the waves. In the situation presented in figure
\ref{fig:rings} all of the test particles have constant spatial
coordinates, they are stationary. However, the measured separation
between the points changes according to $l^2 = -g_{ij} \xi^i
\xi^j$. Modelling this effect via a classical force is questionable.

In order to describe decoherence a master equation approach is
necessary, but the most successful models for dissipation due to a
system's interaction with the environment are not appropriate for the
interaction of a quantum wave function with gravitational waves. The
environment has also been modelled with a quantum field \cite{master}
and for a certain class of interactions the approach is equivalent to
the Caldeira-Leggett model. If a quantum field theory of gravity was
developed then a similar approach could be used to derive a master
equation to model decoherence due to gravitational fields.
