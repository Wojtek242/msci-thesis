\section{Introduction}

A significant amount of experimental effort has gone into detecting
gravitational waves since the 1960s. They were first predicted by
Einstein in 1916 as a consequence of general relativity. Mass (or
energy) warps spacetime and changes in the shape or position of such
objects will cause distortions which propagate as waves at the speed
of light. Gravitational waves have still not been observed
directly. However, the study of the period of the binary pulsar
discovered by Hulse and Taylor in 1974 provides strong indirect
evidence for their existence \cite{pulsar}. The search for direct
evidence of gravitational waves has resulted in a number of
large-scale experiments such as the Laser Interferometer
Gravitational-Wave Observatory (LIGO) and the Laser Interferometer
Space Antenna (LISA). The main difficulty of direct detection of
gravitational radiation is its small effect on a detector, distortions
from equilibrium on Earth due to astrophysical sources are predicted
to be no larger than one part in $10^{21}$ \cite{hobson}. To observe
such a small effect an extremely sensitive apparatus is necessary. The
LIGO and LISA experiments use laser interferometry as a means to
detect such tiny changes. However, the fragile nature of entanglement
could provide an alternative for an experiment to detect this effect.

Entanglement, a phenomenon unique to quantum mechanics, allows for
stronger correlations between separate components of a composite
system than are possible with classical statistics. This ``spooky
action at a distance'' led Einstein to dismiss the theory as an
incomplete description of reality \cite{epr}. However, in 1964 John
Bell showed that no physical theory of local hidden variables, as
suggested by Einstein, can reproduce the predictions of quantum
mechanics. A number of experiments have been performed to test Bell's
theorem and all of them provide strong evidence for the validity of
quantum mechanics. The first definitive experiment was performed by
Alain Aspect in 1982 \cite{aspect}.

Experiments have shown that quantum entanglement is not only real, but
that it can also be used as a resource. The idea that it can be
generated and manipulated like any other physical property of a system
gave rise to the field of quantum information. Entanglement has
allowed us to exceed limits imposed by classical mechanics in
computing, cryptography and data transmission \cite{steane}. However,
in reality quantum entanglement is a fragile resource which is very
difficult to control. Decoherence, the loss of quantum coherences due
to coupling to the environment, occurs on time scales much shorter
than the rate at which we can manipulate the systems experimentally
\cite{zurek}. This sensitivity to environmental effects is one of the
main obstacles in developing quantum technologies and is a subject of
active research. However, this fragility could potentially be used to
measure very small effects that require extremely sensitive
detectors. 

We propose and investigate numerically the possibility of performing
an experiment to detect gravitational waves using the entanglement
between a pair of neutrons initially localized on either side of a
harmonic potential in a multilayer. Entanglement is generated in
collisions due to the particles' natural motion \cite{edmund}. By
working in the weak-field limit of general relativity we combine the
effect of gravitational waves with the Schr\"{o}dinger equation. The
resulting equation is then investigated numerically and we demonstrate
that entanglement amplifies the effect of a gravitational wave, but
the effect is too small to detect using conventional, easily
accessible techniques originally envisaged for this
experiment. However, the results show that entanglement can be a
useful mechanism for detecting high frequency waves.

In section \ref{sec:experiment}, we present the experimental setup
that will be investigated and the mechanism for entanglement
generation. In section \ref{sec:model}, we describe the effect of
gravitational waves, the neutron-neutron interaction and how these
elements are combined in a two-particle Hamiltonian. We also address
various concerns that arise when combining general relativistic
effects with quantum mechanics. Section \ref{sec:numerical} presents
the numerical simulations of the modified Schr\"{o}dinger equation and
their implications for the feasability of the suggested experiment. We
conclude in section \ref{sec:conclusions}.

