\appendix
\section{\large Gravitational Waves in Linearised General Relativity}\label{sec:appendix}

In general relativity we consider pseudo-Riemannian manifolds in which
the interval is given by (assuming the summation convention) 
\begin{equation}
ds^2=g_{\mu\nu}(x)dx^{\mu}dx^{\nu}
\end{equation}
where $g_{\mu\nu}$ are the components of the metric tensor field in
the chosen coordinate system. A weak gravitational field corresponds
to a region of spacetime that is only slightly curved. Thus, there
exist coordinate systems $x^{\mu}$ in which the spacetime metric takes
the form
\begin{equation}\label{eq:linear}
g_{\mu\nu} = \eta_{\mu\nu} + h_{\mu\nu}
\end{equation}
where $|h_{\mu\nu}| \ll 1$, the partial derivatives of $h_{\mu\nu}$
are also small and $\eta_{\mu\nu}$ are the components of the metric
tensor in flat Minkowski spacetime.

It can be shown that for infinitesimal general coordinate
transformations of the form
\begin{equation}
x^{\prime\mu} = x^{\mu}+\xi^{\mu}(x),
\end{equation}
working to first order in small quantities, the metric transforms as
follows:
\begin{equation}
g^{\prime}_{\mu\nu} = \eta_{\mu\nu} + h_{\mu\nu} - \partial_\mu\xi_\nu
- \partial_\nu\xi_\mu.
\end{equation}
We note that $g^{\prime}_{\mu\nu}$ is of the same form
as \eqref{eq:linear}, hence the new metric perturbation functions are
related to the old ones via
\begin{equation}\label{eq:gauge}
h^{\prime}_{\mu\nu} = h_{\mu\nu} - \partial_\mu\xi_\nu
- \partial_\nu\xi_\mu.
\end{equation}

We consider $h_{\mu\nu}$ to be a tensor field defined on the flat
Minkowski background spacetime, hence \eqref{eq:gauge} can be
considered as analogous to a gauge transformation in
electromagnetism. If $h_{\mu\nu}$ is a solution to the linearised
gravitational field equations then the same
physical situation is desribed by \eqref{eq:gauge}. However, in this
viewpoint \eqref{eq:gauge} is viewed as a gauge transformation rather
than a coordinate transformation. We are still working in the same set
of coordinates $x^\mu$ and have defined a new tensor whose components
in this coordinate system are given by \eqref{eq:gauge}. Here we adopt
the viewpoint that $h_{\mu\nu}$ is a symmetric tensor field (under
global Lorentz transformations) defined in quasi-Cartesian coordinates
on a flat Minkowski background spacetime.

The linearised field equations of general relativity can be written as
\begin{equation}
\Box^2\bar{h}^{\mu\nu} = -2\kappa T^{\mu\nu},
\end{equation}
where $T^{\mu\nu}$ is the energy-momentum tensor,
$\bar{h}^{\mu\nu} \equiv h^{\mu\nu} - \frac{1}{2}\eta^{\mu\nu}h$, $h =
h^\mu_\mu$, provided that the $\bar{h}^{\mu\nu}$ components satisfy
the Lorenz gauge condition
\begin{equation}
\partial_\mu\bar{h}^{\mu\nu} = 0.
\end{equation}
In vacuo, where $T^{\mu\nu} = 0$, the general solution of the
linearised field equations may be written as a superposition of
plane-wave solutions of the form
\begin{equation}\label{eq:wave}
\bar{h}^{\mu\nu} = A^{\mu\nu}\exp(ik_\rho x^\rho),
\end{equation}
where $A^{\mu\nu}$ are constant components of a symmetric tensor and
$k_\mu$ are the constant, real components of a vector. The Lorenz gauge
condition is satisfied provided
\begin{equation}
A^{\mu\nu}k_\nu = 0.
\end{equation}
Physical solutions may be obtained by taking the real part
of \eqref{eq:wave}.

Further gauge transformations will preserve the Lorenz gauge condition
provided that the four functions $\xi^\mu(x)$ satisfy $\Box^2\xi^\mu =
0$. A common choice for plane gravitational waves is the
transverse-traceless gauge defined by choosing
\begin{equation}\label{eq:TT}
\bar{h}^{0i}_{TT} \equiv 0 \,\,\,\,\,\, \text{and} \,\,\,\,\,\, \bar{h}_{TT} \equiv 0,
\end{equation}
where latin alphabet indices run over the spatial dimensions
only. Furthermore the Lorenz gauge condition gives the constraints
\begin{equation}\label{eq:lorenzTT}
\partial_0\bar{h}^{00}_{TT} = 0 \,\,\,\,\,\, \text{and} \,\,\,\,\,\, \partial_i\bar{h}^{ij}_{TT} = 0.
\end{equation}
We note that the first condition in \eqref{eq:lorenzTT} implies that
for non-stationary perturbations $\bar{h}^{00}_{TT}$ also
vanishes. Therefore, in this gauge only the spatial components
$\bar{h}^{ij}_{TT}$ are non-zero.

For a particular case of an arbitrary plane gravitational wave of the
form \eqref{eq:wave} the conditions \eqref{eq:TT} imply that
\begin{equation}
A^{0i}_{TT} = 0 \,\,\,\,\,\, \text{and} \,\,\,\,\,\, (A_{TT})^\mu_\mu = 0
\end{equation}
and the Lorenz gauge conditions in \eqref{eq:lorenzTT} require that
\begin{equation}
A^{00}_{TT} = 0 \,\,\,\,\,\, \text{and} \,\,\,\,\,\, A^{ij}_{TT}k_j = 0.
\end{equation}
Under these conditions the $A^{\mu\nu}_{TT}$ components are given by
\begin{equation}
A^{ij}_{TT} = (P^i_kP^j_l - \frac{1}{2}P^{ij}P_{kl})A^{kl},
\end{equation}
where $P_{ij} \equiv \delta_{ij} - n_in_j$ and $n_i = \hat{k}_i$ \cite{hobson}.

\section{The Algorithm}

We solve the two-particle Schr\"{o}dinger equation for the Hamiltonian
\eqref{eq:main} via the explicit staggered method \cite{numerics}. We
choose our units such that $\hbar = 1$ and the neutron mass $m_n =
1$. The wave function is evaluated on a grid of discrete values for
the independent variables
\begin{equation}
\Psi(x_1, x_2, t) = \Psi(x_1 = l\Delta x, x_2 = m\Delta x, t = n\Delta
t) \equiv \Psi^n_{l, m},
\end{equation}
where $l$, $m$ and $n$ are integers. We separate it into real and
imaginary parts,
\begin{equation}
\Psi^n_{l,m} = u^{n+1}_{l,m} + iv^{n+1}_{l,m},
\end{equation}
which allows us to solve the Schr\"{o}dinger equation via a finite
difference method with a pair of coupled equations:
\begin{samepage}
\begin{equation}
u^{n+1}_{l,m} = u^{n-1}_{l,m} - \left\{ \alpha \left[ v^n_{l+1,m} +
  v^n_{l-1,m} + v^n_{l,m+1} + v^n_{l,m-1} - 4v^n_{l,m} \right] - 2
\Delta t V_{l,m}v^n_{l,m} \right\} ,
\end{equation}
\begin{equation}
v^{n+1}_{l,m} = v^{n-1}_{l,m} + \left\{ \alpha \left[ u^n_{l+1,m} +
  u^n_{l-1,m} + u^n_{l,m+1} + u^n_{l,m-1} - 4u^n_{l,m} \right] - 2
\Delta t V_{l,m}u^n_{l,m} \right\},
\end{equation}
\end{samepage}
where 
\begin{equation}
\alpha = g^{xx}\frac{\Delta t}{\Delta x^2},
\end{equation}
and $V_{l,m}$ is the combined value of the potential due to the
harmonic well and neutron-neutron interaction on the discretised
grid given by
\begin{equation}
V_{l,m} = \frac{1}{2}\omega^2(l^2 + m^2)\Delta x^2 + \nu\delta_{l,m}.
\end{equation}
In order to evaluate the value of the real part at step $n+1$ we need
to know the values of the real part at $n-1$ and the imaginary part at
$n$. The same is true for evaluating the imaginary part. Therefore, we
do not have to calculate both parts at every time step and we compute
the real and imaginary parts at slightly different (staggered)
times. The form of the discretised equations is identical to those
presentied in \cite{numerics}. However, the value of $\alpha$ is now
scaled by the oscillating metric tensor component $g^{xx}$. This novel
modification does not lead to instabilities and simulations have been
confirmed to conserve probability to the same precision as before.

For the more general Laplacian \eqref{eq:general}, the equations have to
be modified even further and new terms are introduced for the
wavefunction's first derivative terms present
\begin{equation}
\begin{array}{lcl}
u^{n+1}_{l,m} & = & u^{n-1}_{l,m} - \left\{ \alpha \left[ v^n_{l+1,m}
  + v^n_{l-1,m} + v^n_{l,m+1} + v^n_{l,m-1} -
  4v^n_{l,m} \right] \right. \\\\ & & \left. + \beta \left[ v^n_{l+1,m} -
  v^n_{l-1,m} + v^n_{l, m+1} - v^n_{l, m-1} \right] - 2
\Delta t V_{l,m}v^n_{l,m} \right\} ,
\end{array}
\end{equation}
\begin{equation}
\begin{array}{lcl}
v^{n+1}_{l,m} & = & v^{n-1}_{l,m} + \left\{ \alpha \left[ u^n_{l+1,m}
  + u^n_{l-1,m} + u^n_{l,m+1} + u^n_{l,m-1} -
  4u^n_{l,m} \right] \right. \\\\ & & \left. + \beta \left[
  u^n_{l+1,m} - u^n_{l-1,m} + u^n_{l, m+1} - u^n_{l, m-1} \right] - 2
\Delta t V_{l,m}u^n_{l,m} \right\},
\end{array}
\end{equation}
where
\begin{equation}
\beta = \frac{\partial g^{xx}}{\partial x} \frac{\Delta t}{2 \Delta x}.
\end{equation}
The two components can still be evaluated at staggered times. This
further modification does not cause instabilities and conserves
probability very well.
